\documentclass{article}[12pt]

%\documentclass[global]{svjour}

\renewcommand{\baselinestretch}{1.3}
%\usepackage{showkeys}
\usepackage{amssymb}
\usepackage{amsmath}
\usepackage{enumerate}
\usepackage{latexsym}
\usepackage{mathrsfs}
%\usepackage[small,nohug,heads=vee]{diagrams}
\usepackage{amsthm}
\usepackage{verbatim}
\usepackage{graphicx}
\usepackage{epstopdf}
\usepackage{epsfig}
\usepackage{color}
\usepackage{breqn}
\usepackage{subfigure}
\usepackage{float}
%\usepackage{extarrows}
%\usepackage[algo2e,linesnumbered,ruled]{algorithm2e}
\usepackage{color}
\usepackage{titlesec}
\usepackage{bm}
\usepackage[colorlinks,linkcolor=blue,anchorcolor=green,citecolor=red]{hyperref}
\usepackage[T1]{fontenc}
\renewcommand{\rmdefault}{ptm}
%\usepackage{helvet}
%\usepackage{fourier}
\usepackage{charter}
%\usepackage{DejaVuSansMono}
%\usepackage[expert]{mathdesign}

\usepackage{fancyhdr}
\usepackage{amscd}

\usepackage{caption}
\usepackage{algorithm} %format of the algorithm
\usepackage{algorithmic} %format of the algorithm
\newtheorem{theorem}{Theorem}
\newtheorem{remark}{Remark}
\newtheorem{conclusion}{Conclusion}
\newtheorem{definition}{Definition}
\newtheorem{corollary}{Corollary}
\newtheorem{lemma}{Lemma}
\newtheorem{problem}{Problem}
\newtheorem{proposition}{Proposition}
%\newtheorem{algorithm}{Algorithm}

\makeatletter
%\@addtoreset{equation}{section}
\makeatother

\usepackage{geometry}
\geometry{left=4cm,right=4cm,top=4cm,bottom=4cm}

\begin{document}

\title{Journal Reading Report}
\author{Kyle Chen}
\date{\today}

\maketitle
% \tableofcontents

\section*{DMI infers patterns of synaptic connectivity\cite{endo2015delayed}}
I read an article about application of delayed mutual information on the analysis of synaptic connections.

In this work, experimental and simulating tools are applied to analyze the synaptic connectivities of neuronal network in locust's FeCO. By applying time-delayed mutual information, the time constants and channel capacities of neuronal interaction pathways are measured in terms of delayed time of maximum mutual information and the maximum value of mutual information respectively. In addition, to make up the limitation of mutual information, cross-correlation was adapted to identify the sign of connections.

During the experiment, FeCO sensory neurons are driven by artifitially generated Gaussian white noise(GWN). The synaptic potential of sensory neurons, spiking local interneurons and non-spiking interneurons are recorded, respectively. Time-delayed mutual information between GWN and those three kinds of neurons are calculated. In order to reach 97\% significance level, a surrograte algorithm implemented by Venema et al. (2006) \cite{Venema2006} as a Stochastic Iterative Amplitude Adjustment process was adapted to generate several surragate data based on both experimental and simulating data. With those surragate data, the baseline of mutual information of each neuron pair can be calculated. According to the pattern of mutual information peak of all the neuron pairs tested in experimental, different neuron-neuron connections can be classified. Among the mutual information between GWN and non-spiking neurons, three different types of connections with different delay times and channel capacities which corresponding to three different neuronal pathway discovered in previous related works. Similarly, for spiking local interneurons, there are two kinds of pathways, and for sensory neurons, there is only one kind.

Given different time delays and channel capacities of neuron pairs, non-spiking interneurons can be classified into three clusters based on the Bayesian information criterion. And similar operation was done for spiking local interneurons, which are divided into two clusters.

In addition, based on the time delays of delayed mutual information of target neurons with GWN, the sign of neuronal interaction can be distinguished by checking the sign of cross-correlation at the exact time $t=\tau$.

\begin{thebibliography}{0}
	\bibitem{endo2015delayed}
	Endo, W., Santos, F. P., Simpson, D., Maciel, C. D., \& Newland, P. L. (2015). Delayed mutual information infers patterns of synaptic connectivity in a proprioceptive neural network. Journal of computational neuroscience, 38(2), 427-438.
	\bibitem{Venema2006}
	Venema, V., Ament, F., \& Simmer, C. (2006). A stochastic iterative amplitude adjusted fourier transform algorithm with improved accuracy. Nonlinear Processes in Geophysics, 13(3), 321–328.
\end{thebibliography}

\end{document}
